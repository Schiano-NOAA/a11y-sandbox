\input{accessibility.tex}
% Options for packages loaded elsewhere
% Options for packages loaded elsewhere
\PassOptionsToPackage{unicode}{hyperref}
\PassOptionsToPackage{hyphens}{url}
\PassOptionsToPackage{dvipsnames,svgnames,x11names}{xcolor}
%
\documentclass[
  letterpaper,
]{article}
\usepackage{xcolor}
\usepackage{amsmath,amssymb}
\setcounter{secnumdepth}{5}
\usepackage{iftex}
\ifPDFTeX
  \usepackage[T1]{fontenc}
  \usepackage[utf8]{inputenc}
  \usepackage{textcomp} % provide euro and other symbols
\else % if luatex or xetex
  \usepackage{unicode-math} % this also loads fontspec
  \defaultfontfeatures{Scale=MatchLowercase}
  \defaultfontfeatures[\rmfamily]{Ligatures=TeX,Scale=1}
\fi
\usepackage{lmodern}
\ifPDFTeX\else
  % xetex/luatex font selection
  \setmainfont[]{Latin Modern Sans}
  \setsansfont[]{Latin Modern Sans}
\fi
% Use upquote if available, for straight quotes in verbatim environments
\IfFileExists{upquote.sty}{\usepackage{upquote}}{}
% \IfFileExists{microtype.sty}{% use microtype if available
%   \usepackage[]{microtype}
%   \UseMicrotypeSet[protrusion]{basicmath} % disable protrusion for tt fonts
% }{}
\makeatletter
\@ifundefined{KOMAClassName}{% if non-KOMA class
  \IfFileExists{parskip.sty}{%
    \usepackage{parskip}
  }{% else
    \setlength{\parindent}{0pt}
    \setlength{\parskip}{6pt plus 2pt minus 1pt}}
}{% if KOMA class
  \KOMAoptions{parskip=half}}
\makeatother
% Make \paragraph and \subparagraph free-standing
\makeatletter
\ifx\paragraph\undefined\else
  \let\oldparagraph\paragraph
  \renewcommand{\paragraph}{
    \@ifstar
      \xxxParagraphStar
      \xxxParagraphNoStar
  }
  \newcommand{\xxxParagraphStar}[1]{\oldparagraph*{#1}\mbox{}}
  \newcommand{\xxxParagraphNoStar}[1]{\oldparagraph{#1}\mbox{}}
\fi
\ifx\subparagraph\undefined\else
  \let\oldsubparagraph\subparagraph
  \renewcommand{\subparagraph}{
    \@ifstar
      \xxxSubParagraphStar
      \xxxSubParagraphNoStar
  }
  \newcommand{\xxxSubParagraphStar}[1]{\oldsubparagraph*{#1}\mbox{}}
  \newcommand{\xxxSubParagraphNoStar}[1]{\oldsubparagraph{#1}\mbox{}}
\fi
\makeatother


\usepackage{longtable,booktabs,array}
\usepackage{calc} % for calculating minipage widths
% Correct order of tables after \paragraph or \subparagraph
\usepackage{etoolbox}
\makeatletter
\patchcmd\longtable{\par}{\if@noskipsec\mbox{}\fi\par}{}{}
\makeatother
% Allow footnotes in longtable head/foot
\IfFileExists{footnotehyper.sty}{\usepackage{footnotehyper}}{\usepackage{footnote}}
\makesavenoteenv{longtable}
\usepackage{graphicx}
\makeatletter
\newsavebox\pandoc@box
\newcommand*\pandocbounded[1]{% scales image to fit in text height/width
  \sbox\pandoc@box{#1}%
  \Gscale@div\@tempa{\textheight}{\dimexpr\ht\pandoc@box+\dp\pandoc@box\relax}%
  \Gscale@div\@tempb{\linewidth}{\wd\pandoc@box}%
  \ifdim\@tempb\p@<\@tempa\p@\let\@tempa\@tempb\fi% select the smaller of both
  \ifdim\@tempa\p@<\p@\scalebox{\@tempa}{\usebox\pandoc@box}%
  \else\usebox{\pandoc@box}%
  \fi%
}
% Set default figure placement to htbp
\def\fps@figure{htbp}
\makeatother


% definitions for citeproc citations
\NewDocumentCommand\citeproctext{}{}
\NewDocumentCommand\citeproc{mm}{%
  \begingroup\def\citeproctext{#2}\cite{#1}\endgroup}
\makeatletter
 % allow citations to break across lines
 \let\@cite@ofmt\@firstofone
 % avoid brackets around text for \cite:
 \def\@biblabel#1{}
 \def\@cite#1#2{{#1\if@tempswa , #2\fi}}
\makeatother
\newlength{\cslhangindent}
\setlength{\cslhangindent}{1.5em}
\newlength{\csllabelwidth}
\setlength{\csllabelwidth}{3em}
\newenvironment{CSLReferences}[2] % #1 hanging-indent, #2 entry-spacing
 {\begin{list}{}{%
  \setlength{\itemindent}{0pt}
  \setlength{\leftmargin}{0pt}
  \setlength{\parsep}{0pt}
  % turn on hanging indent if param 1 is 1
  \ifodd #1
   \setlength{\leftmargin}{\cslhangindent}
   \setlength{\itemindent}{-1\cslhangindent}
  \fi
  % set entry spacing
  \setlength{\itemsep}{#2\baselineskip}}}
 {\end{list}}
\usepackage{calc}
\newcommand{\CSLBlock}[1]{\hfill\break\parbox[t]{\linewidth}{\strut\ignorespaces#1\strut}}
\newcommand{\CSLLeftMargin}[1]{\parbox[t]{\csllabelwidth}{\strut#1\strut}}
\newcommand{\CSLRightInline}[1]{\parbox[t]{\linewidth - \csllabelwidth}{\strut#1\strut}}
\newcommand{\CSLIndent}[1]{\hspace{\cslhangindent}#1}



\setlength{\emergencystretch}{3em} % prevent overfull lines

\providecommand{\tightlist}{%
  \setlength{\itemsep}{0pt}\setlength{\parskip}{0pt}}



 


\usepackage{hyphenat}
\usepackage{graphicx}
% and their extensions so you won't have to specify these with
 % every instance of \includegraphics
\usepackage{pdfcomment}
\DeclareGraphicsExtensions{.pdf,.jpeg,.png}
\usepackage{wallpaper} % for the background image on title page
\usepackage{geometry}

\usepackage{lastpage}
% set font
% \usepackage{fontspec}
% \setsansfont{Latin Modern Sans}
% \renewcommand{\setmainfont}[2][]{\fontspec[#1]{Latin Modern Sans}}
% \renewcommand{\rmdefault}{lmss}

% Acronyms
\usepackage[acronym]{glossaries}
\glsdisablehyper
\makenoidxglossaries
\loadglsentries{report_glossary.tex}

% \usepackage{lmodern}
\usepackage[T1]{fontenc}

\newfontfamily\sectionfont[Color=Black]{Latin Modern Sans}
\newfontfamily\subsectionfont[Color=Black]{Latin Modern Sans}
\newfontfamily\subsubsectionfont[Color=Black]{Latin Modern Sans}
% \addtokomafont{section}{\sectionfont}
% \addtokomafont{subsection}{\subsectionfont}
% \addtokomafont{subsubsection}{\subsubsectionfont}
\usepackage[headsepline=0.005pt:,footsepline=0.005pt:,plainfootsepline,automark]{scrlayer-scrpage}
\clearpairofpagestyles
\ohead[]{\headmark} \cofoot[\pagemark]{\pagemark}
\lohead{species assessment 2025}
\ModifyLayer[addvoffset=-.6ex]{scrheadings.foot.above.line}
\ModifyLayer[addvoffset=-.6ex]{plain.scrheadings.foot.above.line}
\setkomafont{pageheadfoot}{\small}

% add soul package to remove latex error
\usepackage{soul}

\usepackage{booktabs}
\usepackage{caption}
\usepackage{longtable}
\usepackage{colortbl}
\usepackage{array}
\usepackage{anyfontsize}
\usepackage{multirow}
\usepackage{wrapfig}
\usepackage{float}
\usepackage{pdflscape}
\usepackage{tabu}
\usepackage{threeparttable}
\usepackage{threeparttablex}
\usepackage[normalem]{ulem}
\usepackage{makecell}
\usepackage{xcolor}
\makeatletter
\@ifpackageloaded{caption}{}{\usepackage{caption}}
\AtBeginDocument{%
\ifdefined\contentsname
  \renewcommand*\contentsname{Table of contents}
\else
  \newcommand\contentsname{Table of contents}
\fi
\ifdefined\listfigurename
  \renewcommand*\listfigurename{List of Figures}
\else
  \newcommand\listfigurename{List of Figures}
\fi
\ifdefined\listtablename
  \renewcommand*\listtablename{List of Tables}
\else
  \newcommand\listtablename{List of Tables}
\fi
\ifdefined\figurename
  \renewcommand*\figurename{Figure}
\else
  \newcommand\figurename{Figure}
\fi
\ifdefined\tablename
  \renewcommand*\tablename{Table}
\else
  \newcommand\tablename{Table}
\fi
}
\@ifpackageloaded{float}{}{\usepackage{float}}
\floatstyle{ruled}
\@ifundefined{c@chapter}{\newfloat{codelisting}{h}{lop}}{\newfloat{codelisting}{h}{lop}[chapter]}
\floatname{codelisting}{Listing}
\newcommand*\listoflistings{\listof{codelisting}{List of Listings}}
\makeatother
\makeatletter
\makeatother
\makeatletter
\@ifpackageloaded{caption}{}{\usepackage{caption}}
\@ifpackageloaded{subcaption}{}{\usepackage{subcaption}}
\makeatother
\usepackage{bookmark}
\IfFileExists{xurl.sty}{\usepackage{xurl}}{} % add URL line breaks if available
\urlstyle{same}
\hypersetup{
  pdftitle={Stock Assessment Report Template},
  pdfauthor={FIRST LAST},
  colorlinks=true,
  linkcolor={blue},
  filecolor={Maroon},
  citecolor={Blue},
  urlcolor={Blue},
  pdfcreator={LaTeX via pandoc}}


\title{Stock Assessment Report Template}
\author{FIRST LAST}
\date{2025-12-11}
\begin{document}
  \begin{titlepage}
  % This is a combination of Pandoc templating and LaTeX
  % Pandoc templating https://pandoc.org/MANUAL.html#templates
  % See the README for help

  \newgeometry{top=2in,bottom=1in,right=1in,left=1in}
  \begin{minipage}[b][\textheight][s]{\textwidth}
  % Ross would've subbed lines 6, 8 with these lines:
  %\newgeometry{top=2in,bottom=1in,right=1in,left=1in}%
  %\noindent  %\tracingall
  %\begin{minipage}[b][\textheight][s]{.975\textwidth}%% RRM: avoid Overfull box


  \raggedright

  % \includegraphics[width=2cm]{NOAA_Transparent_Logo.png}

  % background image


  % Title and subtitle
  {\huge\bfseries\nohyphens{Stock Assessment Report
  Template}}\\[1\baselineskip]
  % Ross would change the end of the above line to the following because \par must come before the group closes and line-depth reverts.
  % }\par}%\\[1\baselineskip]



  \vspace{2\baselineskip}
  % Ross would change this to 2\baselineskip

  %%%%%% Cover image

  \vspace{1\baselineskip}

  % Authors
  % This hairy bit of code is just to get "and" between the last 2
  % authors. See below if you don't need that
  %
  {\large{FIRST LAST}}%
  %
  {\textsuperscript{1}}%
  %


  % This is how to do it if you don't need the "and"

  %%%%%% Affiliations
  \vspace{2\baselineskip}

  \hangindent=1em
  \hangafter=1
  % Ross would change the above line to:
  % \hangafter=1\relax
  %
  {1}.~{NOAA Fisheries}%
  %
  %
  % Ross recommends putting address on one line
  , %
  % {ADDRESS}%
  {ADDRESS, CITY, POSTAL CODE}%
  %


  %%%%%% Correspondence
  \vspace{1\baselineskip}


  %use \vfill instead to get the space to fill flexibly
  %\vspace{0.25\textheight} % Whitespace between the title block and the publisher

  \vfill


  % Whitespace between the title block and the tagline
  \vspace{1\baselineskip}

  %%%%%% Tagline at bottom
  % Ross says the tagline below could also be centered
  \includegraphics[width=2cm, artifact]{support_files/us_doc_logo.png}\newline % empty curly brackets without alt text is suitable for this logo because it's purely decorative/an "artifact"
  U.S. Department of Commerce\newline
  National Oceanic and Atmospheric Administration\newline
  National Marine Fisheries Service\newline

  \end{minipage}
  \restoregeometry
  \end{titlepage}

\newgeometry{top=1in,bottom=1in,right=1in,left=1in}

\pdfbookmark[1]{Table of Contents}{toc}
\renewcommand*\contentsname{Table of contents}
{
\hypersetup{linkcolor=.}
\setcounter{tocdepth}{3}
\tableofcontents
}
\listoffigures
\listoftables

\printnoidxglossaries

\newpage{}

\subsection*{Disclaimer}\label{disclaimer}

These materials do not constitute a formal publication and are for
information only. They are in a pre-review, pre-decisional state and
should not be formally cited or reproduced. They are to be considered
provisional and do not represent any determination or policy of NOAA or
the Department of Commerce.

\newpage{}

Please cite this publication as:

{[}AUTHOR NAME{]}. {[}YEAR{]}. Stock Assessment Report Template.
National Marine Fisheries Service, {[}CITY{]}, {[}STATE{]}.
\pageref*{LastPage}{} pp.

\newpage{}

\section{Executive Summary}\label{sec-exec-sum}

\subsection{Assessment Model}\label{assessment-model}

\subsection{Reference Points, Stock Status, and
Projections}\label{reference-points-stock-status-and-projections}

\gls{abc} \gls{nwfsc}

\newpage{}

\section{Introduction}\label{sec-intro}

Testing adding in an introduction for species. There is currently no
read of parameters for child documents.

\subsection{Stock ID}\label{stock-id}

\subsection{Management History}\label{management-history}

\subsection{Fishery Descriptions}\label{fishery-descriptions}

\subsection{Ecosystem Considerations}\label{ecosystem-considerations}

Ecosystem considerations and/or climate indicators were not included in
this assessment.

\newpage{}

\section{Data}\label{sec-data}

\subsection{Life History}\label{life-history}

\subsection{Catch}\label{catch}

\subsection{Indices and
Standardization}\label{indices-and-standardization}

\subsection{Composition Data}\label{composition-data}

\subsection{Absolute Abundance}\label{absolute-abundance}

\subsection{Environmental/Ecosystem Indicator
Data}\label{environmentalecosystem-indicator-data}

\newpage{}

\section{Assessment}\label{sec-assmt-config}

\subsection{Current Modeling Approach}\label{current-modeling-approach}

\subsection{Configuration of the Base
Model}\label{configuration-of-the-base-model}

\subsection{Bridging}\label{bridging}

\newpage{}

\subsection{Modeling Results}\label{sec-assmt-results}

\subsubsection{Parameter Estimates}\label{parameter-estimates}

\subsubsection{Time Series}\label{time-series}

\subsubsection{Model Fits}\label{model-fits}

\subsubsection{Model Diagnostics}\label{model-diagnostics}

\newpage{}

\subsection{Sensitivity Analyses}\label{sec-assmt-sens}

\newpage{}

\subsection{Management Benchmarks}\label{sec-assmt-bench}

\newpage{}

\subsection{Projections}\label{sec-assmt-proj}

\newpage{}

\section{Discussion}\label{sec-discussion}

\newpage{}

\section{Acknowledgements}\label{sec-acknowledgements}

This document was produced using the R package asar (Schiano et al.
2025), which is free to use and publicly available on
\href{https://github.com/nmfs-ost/asar}{GitHub}.

\newpage{}

\section{References}\label{sec-refs}

\phantomsection\label{refs}
\begin{CSLReferences}{1}{0}
\bibitem[\citeproctext]{ref-asar_2025}
Schiano, S., Breitbart, S., and Saul, S. 2025. Asar: Build NOAA stock
assessment report. Available from
\url{https://github.com/nmfs-ost/asar}.

\end{CSLReferences}

\newpage{}

\section{Tables}\label{sec-tables}

\tagpdfsetup{table/header-rows={1}}
\begin{table}[h!]

\caption{\label{tbl-landgt}This is my cool caption for a gt table.}

\centering{

\fontsize{12.0pt}{14.4pt}\selectfont
\begin{tabular*}{\linewidth}{@{\extracolsep{\fill}}lrrl}
\toprule
label & year & estimate & fleet \\ 
\midrule\addlinespace[2.5pt]
landings\_observed\_weight & NA & 0 & North \\ 
landings\_predicted\_weight & NA & 0 & North \\ 
landings\_observed\_weight & 1876 & 0 & North \\ 
landings\_predicted\_weight & 1876 & 0 & North \\ 
landings\_observed\_weight & 1877 & 0 & North \\ 
landings\_predicted\_weight & 1877 & 0 & North \\ 
\bottomrule
\end{tabular*}

}

\end{table}%
\tagpdfsetup{table/header-rows={1}}

\begin{longtable}[]{@{}lrrl@{}}

\caption{\label{tbl-landkable}This is my cool caption for a kable
table.}

\tabularnewline

\toprule\noalign{}
label & year & estimate & fleet \\
\midrule\noalign{}
\endhead
\bottomrule\noalign{}
\endlastfoot
landings\_observed\_weight & NA & 0 & North \\
landings\_predicted\_weight & NA & 0 & North \\
landings\_observed\_weight & 1876 & 0 & North \\
landings\_predicted\_weight & 1876 & 0 & North \\
landings\_observed\_weight & 1877 & 0 & North \\
landings\_predicted\_weight & 1877 & 0 & North \\

\tagpdfsetup{table/header-rows={1}}
\end{longtable}

\begin{table}[h!]

\caption{\label{tbl-landkbl}This is my cool caption for a kbl table.}

\centering{

\begin{tabular}[t]{l|r|r|l}
\hline
label & year & estimate & fleet\\
\hline
landings\_observed\_weight & NA & 0 & North\\
\hline
landings\_predicted\_weight & NA & 0 & North\\
\hline
landings\_observed\_weight & 1876 & 0 & North\\
\hline
landings\_predicted\_weight & 1876 & 0 & North\\
\hline
landings\_observed\_weight & 1877 & 0 & North\\
\hline
landings\_predicted\_weight & 1877 & 0 & North\\
\hline
\end{tabular}

}

\end{table}%

\newpage{}

\section{Figures}\label{sec-figures}

\begin{figure}

\centering{

\pandocbounded{\includegraphics[keepaspectratio,alt={'Line graph showing biomass time series. The x axis shows the year, which spans from 1874 to 2022. The y axis shows biomass in mt, which spans from 5420.45 to 42648.7.'}]{SAR_species_skeleton_files/figure-pdf/fig-biomass-1.png}}

}

\caption{\label{fig-biomass}Biomass (B) time series. The horizontal
dashed line represents the limit reference point (msy mt).}

\end{figure}%

\begin{figure}

\centering{

\pandocbounded{\includegraphics[keepaspectratio,alt={'Bubble plot showing model-estimated population numbers at age and population biomass at age. The x axis shows the year, which spans from 1874 to 2022 . The y axis shows age, which spans from 0 to 40. The size of the bubbles range from 0 to 42648.7.'}]{SAR_species_skeleton_files/figure-pdf/fig-pop.baa-1.png}}

}

\caption{\label{fig-pop.baa}Model-estimated population numbers at age
and population biomass at age over time. The relative size of each
bubble for a given year and age indicates the relative abundance or
biomass in that category compared with others.}

\end{figure}%

\begin{figure}

\centering{

\pandocbounded{\includegraphics[keepaspectratio,alt={'Bubble plot showing model-estimated total catch at age. The x axis shows the year, which spans from 1874 to 2034 . The y axis shows age, which spans from 0 to 40. The size of the bubbles range from NA to NA.'}]{SAR_species_skeleton_files/figure-pdf/fig-pop.caa-1.png}}

}

\caption{\label{fig-pop.caa}Fishery age composition (1874-2034). The
area of the circle is proportional to the catch. Diagonal lines
indicated the top 5\% strongest year classes.}

\end{figure}%

\begin{figure}

\centering{

\pandocbounded{\includegraphics[keepaspectratio,alt={'Bubble plot showing relative age proportions over time. The x axis shows the year, which spans from 1874 to 2022 . The y axis shows age in years, which spans from 0 to 40. The size of the bubbles range from 0 to 37952.5.'}]{SAR_species_skeleton_files/figure-pdf/fig-pop.naa-1.png}}

}

\caption{\label{fig-pop.naa}Model estimate of population numbers at age
over time. The relative size of each bubble for a given year and age
indicates the relative abundance in that category compared with others.}

\end{figure}%

\begin{figure}

\centering{

\pandocbounded{\includegraphics[keepaspectratio,alt={'Scatterplot showing annual deviations in recruitment. Points have error bars and the dashed horizontal line at 0 represents no deviation from what would be estimated by the stock-recruit relationship. Positive values represent an increase in recruitment that year while negative values represent a decrease. The x axis shows year, which spans from 1845 to 2022. The y axis shows the recruitment deviation, which spans from -0.62 to 1.15 on a natural log scale.'}]{SAR_species_skeleton_files/figure-pdf/fig-recruitment.deviations-1.png}}

}

\caption{\label{fig-recruitment.deviations}Annual deviations (on natural
log scale) in the number of newly recruited fish the model estimates
each year.}

\end{figure}%

\begin{figure}

\centering{

\pandocbounded{\includegraphics[keepaspectratio,alt={'Line graph showing model-estimated spawning stock biomass. The x axis shows the year, which spans from 1876 to 2022. The y axis shows model-estimated spawning stock biomass in mt, which spans from 1.3 to 15357.1.'}]{SAR_species_skeleton_files/figure-pdf/fig-spawning.biomass-1.png}}

}

\caption{\label{fig-spawning.biomass}Model-estimated spawning stock
biomass (SSB) time series. The horizontal dashed line represents the
spawning stock biomass associated with the biomass limit reference point
(msy mt).}

\end{figure}%

\begin{figure}

\centering{

\pandocbounded{\includegraphics[keepaspectratio,alt={'Alt text for an external image exampe'}]{plot.png}}

}

\caption{\label{fig-external}External image that already has a png
extension}

\end{figure}%

\newpage{}

\appendix
\setcounter{figure}{0}
\setcounter{table}{0}

\section{Appendices}\label{sec-appendix}




\end{document}
